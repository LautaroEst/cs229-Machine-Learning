\section{Parte 2: Aprendizaje Estadístico}

Una definición satisfactoria de ``aprendizaje'' es la dada por Mitchell (1997), y es la que se toma como punto de partida para diseñar cualquier algoritmo de ML:

``Se dice que un programa que realiza una tarea T y tiene una medida de desempeño P aprendió de una experiancia E, si su desempeño en la tarea T mejoró con la experiencia E.''

Esta definición tiene sus limitaciones (como todo), pero es la que permite generalizar la manera en que se construye o se diseña un algoritmo de ML.
