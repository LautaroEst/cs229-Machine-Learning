\item \subquestionpoints{10} Consider a linear regression problem in which
we want to ``weight'' different training examples differently.  Specifically,
suppose we want to minimize
%
\begin{equation*}
	J(\theta) = \frac{1}{2} \sum_{i=1}^m w^{(i)}
		\left(\theta^Tx^{(i)} - y^{(i)}\right)^2.
\end{equation*}
%
In class, we worked out what happens for the case where all the weights (the
$w^{(i)}$'s) are the same. In this problem, we will generalize some of those
ideas to the weighted setting.
\begin{enumerate}
	\item \subquestionpoints{2} Show that $J(\theta)$ can also be written
    %
    \begin{equation*}
    J(\theta) = (X\theta - {y})^T W (X\theta - {y})
    \end{equation*}
    %
    for an appropriate matrix $W$, and where $X$ and ${y}$ are as
    defined in class. Clearly specify the value of each element of the matrix
    $W$.

	\item \label{item:lwr-solution} \subquestionpoints{4} If all the $w^{(i)}$'s
    equal 1, then we saw in class that the normal equation is
    %
    \begin{equation*}
    X^TX\theta = X^T{y},
    \end{equation*}
    %
	and that the value of $\theta$ that minimizes $J(\theta)$ is given by
	$(X^TX)^{-1}X^T{y}.$
	By finding the derivative $\nabla_\theta J(\theta)$ and setting that to zero,
	generalize
	the normal equation to this weighted setting, and give the new value of
	$\theta$ that minimizes
    $J(\theta)$ in closed form as a function of $X$, $W$ and ${y}$.

	\item \subquestionpoints{4} Suppose we have a dataset
	$\{(x^{(i)}, y^{(i)});\, i=1\ldots,m\}$ of $m$ independent examples, but
    we model the $y^{(i)}$'s as drawn from conditional distributions with
    different levels of variance $(\sigma^{(i)})^2$. Specifically, assume the
    model
    %
    \begin{equation*}
		p(y^{(i)} | x^{(i)} ; \theta) = \frac{1}{\sqrt{2\pi}\sigma^{(i)}} \exp\left(-
		\frac{(y^{(i)} - \theta^Tx^{(i)})^2}{2(\sigma^{(i)})^2}\right)
	\end{equation*}
    %
    That is, each $y^{(i)}$ is drawn from a Gaussian distribution with mean
    $\theta^Tx^{(i)}$ and variance $(\sigma^{(i)})^2$ (where the
    $\sigma^{(i)}$'s are fixed, known, constants). Show that finding the
    maximum likelihood estimate of $\theta$ reduces to solving a weighted
	linear regression problem.  State clearly what the $w^{(i)}$'s are in terms of
    the $\sigma^{(i)}$'s.
\end{enumerate}

\ifnum\solutions=1
  \begin{answer}
    \begin{enumerate}
    \item
    We have

    $$
    \begin{aligned}
J(\theta) &= \sum_{i=1}^m (\theta^Tx^{(i)} - y^{(i)})(\frac{1}{2}w^{(i)}) (\theta^Tx^{(i)} - y^{(i)})\\
&= \sum_{i=1}^m (X\theta -y)^T_i (\frac{1}{2}w^{(i)}) (X\theta- y)_i
\end{aligned}
$$

If we let 

$$
W_{ij} = \begin{cases}
\frac{1}{2}w^{(i)} &i=j\\
0&\text{otherwise}
\end{cases}
$$

Then 

$$
\begin{aligned}
J(\theta)
&= \sum_{i=1}^m (X\theta -y)^T_i (\frac{1}{2}w^{(i)}) (X\theta- y)_i\\
&= \sum_{i, j=1}^m (X\theta -y)^T_i W_{ij} (X\theta- y)_j\\\\
&= (X\theta - y)^T W(X\theta - y)
\end{aligned}
$$
\item
    We have

    $$
    \begin{aligned}
\frac{\partial}{\partial \theta}J(\theta) &= \frac{\partial}{\partial \theta} (\theta^TX^TWX\theta - 2y^TW X\theta + y^TWy)\\
&=  2X^TWX\theta - 2 X^TW^Ty
\end{aligned}
$$

Taking the derivatives to zero, we find

$$
\theta = (X^TWX)^{-1}X^TW y
$$

\item

The log likelihood

$$
\begin{aligned}
l(\theta) &= \sum_{i=1}^m \log p(y^{(i)}|x^{(i)};\theta)\\
&= \sum_{i=1}^m(-\log \sqrt2\pi - \log \sigma^{(i)} - \frac{1}{2(\sigma^{(i)})^2} (y^{(i)} - \theta^Tx^{(i)})^2)
\end{aligned}
$$

Maximizing this is equivalent to minimizing

$$
            J(\theta) = \sum_{i=1}^m\frac{1}{2(\sigma^{(i)})^2} (y^{(i)} - \theta^Tx^{(i)})^2)
            $$

    \end{enumerate}

So here $w^{(i)} = 1/ (\sigma^{(i)})^2$

\end{answer}

\fi
